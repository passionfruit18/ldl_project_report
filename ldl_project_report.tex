%%% basic stuff
\documentclass[11pt, oneside]{article}
\usepackage{geometry}
\geometry{letterpaper}

%%% Verbatim packages
\usepackage{verbatim}
\usepackage{fancyvrb}
\DefineVerbatimEnvironment{code}{Verbatim}{fontsize=\small}

\begin{comment}
%%% Todo:

> Write about LDL.
>> Find out how to write context free grammars.
>> Find out how to draw diagrams.
> Modularise this document.
> Share this document.
> Link this document with the code somehow. Make each findable online from each other.

\end{comment}

\begin{comment}
    %%% LaTeX maths codes 10 Oct

    \mathcal{} � math calligraphic
    font
    \lor, \land, \to , \neg� logical or, logical and, right arrow, negation
    \subseteq � subset or equal
    \subset � strict subset
    \supset, \supseteq
    \in � element of
    \forall, \exists
    \Box, \Diamond � modal operators
    \prime
    \ell � cursive l
    \models � entails
    \Rightarrow implies
    \bigcap�{i=1}^{n}� big intersection
    \bigcup ""� big union
    \bigwedge, \bigvee� big and, big or

    { and } must be escaped
    %%% LaTeX general codes
    \textbf{} = bold
    \newcommand {\comm}{definition of comm}
    \newcommand {\comm}[n]{some function of e(#1)...e(#n)}


    %%% math spacing

    \quad
    \, 3/18
    \: 4/18
    \; 5/18
    \! -3/18
    \  (space after \) normal space
\end{comment}

%%% Fonts
\usepackage{lmodern}

%%% Citation
\usepackage{cite}

%%% LTL and LDL Syntax; ``definition'' apparatus
\usepackage{bnf}
\newcommand{\semi}{; } % since ; is a special character for the bnf package
\newcommand{\model}{\mathfrak{M} } % letter for "model"
\newenvironment{myGrammar} % The grammar formmating I'll be using
    {   \begin{grammar}
        [(colon){ ::=}]
        [(semicolon){ $|$}]
    }
    {   \end{grammar}}
\newenvironment{myGrammarPlus} % The same, but with a +.
    {   \begin{grammar}
        [(colon){ +::=}]
        [(semicolon){ $|$}]
    }
    {   \end{grammar}}

%%% LTL and LDL syntax

\let\vphi\varphi % alternate phi, which I'll be using a lot
\newcommand{\di}[1]{\langle#1\rangle} % 'diamond' For LDL modalities
\newcommand{\sq}[1]{[#1]} % 'square'
\newcommand{\luntil}{\ \mathcal{U}\ }
\newcommand{\lalways}{\mathcal{G}}
\newcommand{\leventually}{\mathcal{F}}
\newcommand{\lnext}{\circ\,}
\newcommand{\prog}{\Pi}
\newcommand{\A}{\mathnormal{A}}
\newcommand{\prop}{\mathcal{P}}
\newcommand{\actions}{\mathcal{A}}
\newcommand{\fin}[1]{#1$_f$}
\newcommand{\ltlf}{LTL$_f\,$}
\newcommand{\ldlf}{LDL$_f\,$}

\newcommand{\red}[1]{{\color{red} #1}}

%%% Code listings
\usepackage{framed}

%%% Diagram drawing
\usepackage{tikz}
\usetikzlibrary{arrows,automata,calc,shapes, positioning}

%%% Math packages
\usepackage{amsmath}
\usepackage{amsthm} %theorem
\usepackage{amssymb} %symbol
\usepackage[inference, reserved]{semantic} % for inference rules and programming keywords
\reservestyle[\ ]{\command}{\mathbf}
\command{if,then,else,while,do}

%% For defining Lemmas and Definitions
\newtheorem{lem}{Lemma}[section]
\newtheorem{defn}{Definition}[section]
\newcommand{\defnWithLabel}[2]{\begin{defn} \label{defn:#1} #2 \end{defn}}
\newenvironment{defnL}[1]{\begin{defn} \label{defn:#1}}{\end{defn}}

%%% For defining runs of automata
\newcommand{\rundef}[3]{\begin{itemize} \item \condition{Initial} #1 \item \condition{Transition} #2 \item \condition{Accepting} #3 \end{itemize}}
\newcommand{\condition}[1]{(#1 Condition):}

%%% Some notation:
\newcommand{\powerset}[1]{\mathcal{P}(#1)}
\newcommand{\posbool}[1]{B^{+}(#1)} % Positive boolean formulae
\newcommand{\negbool}[1]{B^{-}(#1)}
\newcommand{\bool}[1]{B(#1)}
\newcommand{\set}[1]{\{#1\}}
\newcommand{\evalto}{\xrightarrow{eval}} % 'evaluates to'
\newcommand{\nonempty}[1]{#1 \backslash \emptyset}

\def\Xrightarrow#1#2{\xrightarrow#1{}\negthickspace^#2} % For writing transitive closures e.g. $ \Xrightarrow\tau* $
\newcommand{\epsilonReaches}{\Xrightarrow\epsilon*} % 'reaches by 0 or more epsilon-transitions'

%%% Hyperlink References
\usepackage[colorlinks=true]{hyper ref}

\title{LTL and LDL on Finite Traces: Exposition} \author{James Dai}
%\date{}

\begin{document}
\maketitle \tableofcontents \newpage
\section{Abstract}
\begin{center}
  \red{\emph{In which I reverse-engineer a rather tersely-written paper}}
\end{center}
\section {Introduction}
\begin{comment}
I mainly follow the arguments of Vardi and de Giacomo in the paper ``Linear Temporal Logic and Linear Dynamic Logic on Finite Traces'' \cite{ldlf},
as well as parts of their other papers in the same area.
A wide range of ideas are woven into this setting. Planning, processes, Dynamic Logic, alternating automata.
I also implement a satisfiability algorithm, though this is somewhat an afterthought. Or is it?? Well.
It is a bit of a makeshift thing to be honest.
\end{comment}
\red{1. The aim of this report} is to explore, explain, and implement the concepts in the paper
``Linear Temporal Logic and Linear Dynamic Logic on Finite Traces'' by Giuseppe De Giacomo and Moshe Y. Vardi \cite{ldlf},
from an undergraduate's perspective.
\red{2. I aim to present the formalisms} of \ltlf, \ldlf and AFAs and their connexions in a way
that can be understood by an undergraduate in computer science.
\red{3. The aim of this report is} to discuss LTL in a relatively new setting, that of finite traces.
The paper discusses Linear Temporal Logic (LTL),
which is usually interpreted over infinite traces, over finite traces.
To distinguish the two, one is written LTL, and the other \ltlf.
There are many uses of \ltlf in artificial intelligence.
The paper shows that \ltlf is less expressive than regular expressions over finite traces,
and introduces a logic, Linear Dynamic Logic, that is as expressive as regular expressions,
yet as intuitive as linear temporal logic, and has the same computational properties (i.e. satisfiability is PSPACE-complete).
I present some examples of the uses of \ltlf in planning,
give examples of where it is not expressive enough,
and cover the algorithm given in \cite{ldlf} to decide the satisfiability of \ldlf,
which goes via ``alternating automata''. I have implemented this algorithm in Haskell.



\section{Linear Temporal Logic on Finite Traces: Uses and Limitations}

\subsection{LTL on infinite traces: A logic for specifying behaviours of infinite processes.}
Historically LTL has been used in the infinite setting,
In the infinite setting LTL is concerned with concurrent processes which run indefinitely. By constraints on 'traces' of the observable/recorded behaviours of systems of these interacting processes, it can ensure/verify properties such as:

Safety:
Mutual exclusion (two processes don't access a resource
(e.g. a printer) at the same time).
\begin{center}
  $\lalways(using_1 \to \neg using_2 \luntil \neg using_1)$
\end{center}
says one direction of mutual exclusion,
viz. while one process is using the printer ($using_1$ is true)
the other may not ($using_2$ is false).

Fairness:
After making a request (to use a printer) it is always fulfilled (eventually)
and not blocked up by another user indefinitely.
\begin{center}
  $\lalways(req_1 \to \leventually using_1)$


Liveness:
The most basic liveness is, ``X happens''. So, $\leventually a$ says proposition $a$ will hold eventually.
Or, infinitely often: $\lalways \leventually a$. (modalities associate to the right.)
\subsection{Syntax and semantics}
\red{STUB}
The syntax of LTL is propositional logic
(i.e. formulae can be in the base case propositions $a$,
joined by connectives $\lor$ and $\neg$),
over a given set of propositions
$\prop$ (e.g. $\set{a,b,c}$) with modalities ``next'' and ``until'', written
$\lnext \vphi$ and $\vphi \luntil \psi$.
\\The syntactic abbreviations
$\leventually \vphi := true \luntil \vphi$ and
$\lalways \vphi := \neg \leventually \neg \vphi$,
have a semantics of ``eventually'' (or ``Future'')
and ``always'' (or ``Globally'').
\\A (infinite) model for LTL is an infinite sequence
$\pi$ (also called a trace) of propositional models (represented as sets) (see figure \ref{fig:inf_trace_diagram}).

{\newcommand{\lab}{$\lnext$}
\begin{figure}
\begin{tikzpicture}[shorten >=1pt,node distance=3cm,on grid,auto]
   \node[state] (q1)   {$\pi[0]: \set{a}$};
   \node[state] (q2) [right=of q1] {$\pi[1]: \set{a,b}$};
   \node[state] (q3) [right=of q2] {$\pi[2]: \set{}$};
   \node[state, draw=none] (q4) [right=of q3] {...};
    \path[->]
    (q1) edge node {\lab} (q2)
    (q2) edge node {\lab} (q3)
    (q3) edge node {\lab} (q4);

\end{tikzpicture}
    \caption{Diagram of an infinite trace\label{fig:inf_trace_diagram}}
\end{figure}
}


\\A formula $\vphi$ is satisfied by trace $\pi$ if it is satisfied
from the beginning of the trace, that is,
we say $\pi \models \vphi$ iff $\pi, 0 \models \vphi$.
(Note this is not the same as the
propositional world $\pi[0]$ satisfying $\vphi$!)
We have the modal conditions
\\
\begin{itemize}
\item$\pi, i \models \lnext\vphi$ iff $\pi, (i+1) \models\vphi$
\item $\pi, i \models \vphi\luntil\psi$ iff
either $\pi, i \models \psi$ or
($\pi, i \models \vphi$ and $\pi, (i+1) \models \vphi\luntil\psi$).
\end{itemize}
Equivalently, we can say $\vphi\luntil\psi \equiv
\psi \lor (\vphi \land \vphi\luntil\psi)$.
Or, giving a non-recursive definition,
$\pi, i \models \vphi \luntil \psi$ iff
$(\exists j \geq i)
(\pi,j \models \psi \land
(\forall i \leq k < j) (\pi, k \models \vphi))$

Otherwise, $\lor$ and $\neg$ behave as expected, and propositions $a \in \prop$ are evaluated
$\pi,i \models a$ iff $\pi[i] \models a$.
\\Models for \ltlf are finite sequences $\pi[0]..\pi[last]$ (see figure \ref{fig:fin_trace_diagram}).
What happens at the edge.
Note that for no $\vphi$ does $\pi[last] \models \lnext\vphi$.
\red{Stuff mentioned in paper.}
Certain formulae equivalent. $\lalways \leventually \vphi$ and
$\leventually \lalways \vphi$ are satisfied by and only by $\pi$ such that $\pi, last \models \vphi$.

{\newcommand{\lab}{$\lnext$}
\begin{figure}
  \begin{tikzpicture}[shorten >=1pt,node distance=3cm,on grid,auto]
     \node[state] (q1)   {$\pi[0]: \set{a}$};
     \node[state] (q2) [right=of q1] {$\pi[1]: \set{a,b}$};
     \node[state] (q3) [right=of q2] {$\pi[2]: \set{}$};
     \node[state, draw=none] (q4) [right=of q3] {...};
     \node[state] (q5) [right=of q4] {$\pi[last]: \set{c}$};
      \path[->]
      (q1) edge node {\lab} (q2)
      (q2) edge node {\lab} (q3)
      (q3) edge node {\lab} (q4)
      (q4) edge node {\lab} (q5);
  \end{tikzpicture}
\caption{Diagram of a finite trace\label{fig:fin_trace_diagram}}
\end{figure}
}


\subsection{Applications in the finite setting: planning and service process modelling}
\red{STUB}

Having considered in abstractum the differences of finite and infinite setting, we shall briefly discuss some applications of \ltlf.
One is ``one step down'' from the infinite process behaviour specification: of course, it is ``finite process behaviour specification'', which takes its major form in
``Business Process Modelling'', in which some service is to be rendered to a client, in the process of which many different agents will be involved
(hence this has some of the ``concurrent'' flavour of the infinite process case.)
The major formalism in this area is called DECLARE (see e.g. \cite{decserflow}), which ``compiles'' into \ltlf.

On the other hand, there is what seems to be an entirely different use, which is as a way to encode planning problems.
A formula is interpreted as a problem, and a trace as a possible plan solving the problem.
This involves a little bit of special encoding (which would be factored into the compilation from e.g. the major planning formalism,
PDDL 3.0, to \ltlf). This application is important so it will be worth understanding a little bit.

By nature a formula only talks about propositions that hold at any given ``state'' in a trace.
But planning about taking \emph{actions} to get from one state to another (at which certain conditions hold, which can be encoded as propositions).
We can think of this contrast as between \emph{node labels} and \emph{edge labels}
(visualising a trace as a sequence of nodes (states $\pi[i]$), and directed edges from each $\pi[i]$ (except the last) to the successor $\pi[i+1]$).
LTL does not have a notion of actions or edge labels.
Thus actions are encoded as propositions (which are after all the only things available to use).
There is a special set of propositions $\actions$, disjoint from the ordinary propositions $\prop$,
and certain conditions are imposed on them that any trace will have to satisfy, viz.
1. exactly one action proposition must hold at each state except the first (at which none),
representing the action taken to get to that state,
and 2. the action proposition at a state $\pi[i]$
and the other propositions from $\prop$ at $\pi[i-1]$ and $\pi[i]$ must
``behave properly'' according to the planning problem definition (that stipulate the conditions, encoded as $\prop$, which must hold before and after an action is taken).

More on some notions which arise naturally in the settings mentioned above,
and their representation in their respective formalisms and thus in LTL.

The formalisms associated with planning and (terminating) process/service modelling are
PDDL 3.0.
In the DECLARE formalism are defined certain commonly used patterns such as:
Response, Succession, Precedence, Chain, Existence, ...
(in fact these are represented graphically)
which are encoded in \ltlf as:


\red{tangent?}
There is quite a strong connection between \ltlf and LTL on infinite traces,
since the question of whether a model satisfies an \ltlf formula can,
in many cases, be directly answered by using tools originally developed for LTL formulae on
\emph{infinite} traces (otherwise, the formula would have to be translated).
The formulae which can be thus handled are called `insensitive to infiniteness'.
\cite{ldlfinsens}
Many common patterns captured by \ltlf are such formulae.

e.g.

\subsection{variations on goals}
\red{STUB}
\ltlf can also capture temporal constraints while trying to reach the goal.
It can also capture temporally extended goals.
For example, ``do not disturb'' or ``restore'' patterns are like safety patterns:
They say ``while pursuing this goal, maintain some integrity''. e.g. don't harm people.
Or ``clean your dishes after you use them'' (returning things to a previous clean state).
Or ``if you do hurt someone, apologize''...

\subsection{???}
\red{STUB}
logics can be used in a variety of ways. As we have been saying,
they can encode planning problems, and
we can decide whether a plan solves a problem, \emph{a posteriori}, that is,
with the plan already having been made.
Or, they can be used to \emph{monitor} an \emph{ongoing} (business) process, saying
what conditions need to hold (be restored) for the process to finish, and also
giving some indication of what actions should \emph{not} be taken.
We could think of this being implemented by a translation to an automaton \emph{on the fly},
so that not all states have to be generated at once.
Or again, similarly, if we have a particular plan we want to evaluate, we don't have to
realize the whole automaton, but simply the part relevant to the plan actions.

\subsection{(Expressive) Limitations of \ltlf}
Procedural execution constraints cannot be expressed in \ltlf.
\cite{ldlfsynth} (i.e. programs)

Especially in the context of planning, we see a limitation when we want to express
``procedural execution constraints''. Essentially, when we want to have more structure in our actions;
when we want to have `compound actions', which are basically programs.
And this ties in very nicely with \ldlf since it is based on PDL which intrinsically has a notion of programs.
Although it's a bit awkward because \ldlf mimcs LTL in not really having edge labels, or rather it
gives up its edge labels for a sort of backwards-compatability with \ltlf, but then we need edge labels, so
we must go the old route of encoding actions as propositions...

Well, I suppose there are advantages to doing things this way, but I can't help but wonder
what would happen if a more direct representation of actions was allowed...
Well, one would want to keep linearity of the atomic actions,
and then the only question is whether the thing still compiles nicely into AFAs?

\red{So, anyways, we need a new formalism. Hence \ldlf.}

\ldlf has the same complexity for satisfiability etc., but is more expressive. It can handle those things mentioned above. (Is the same true on infinite?)

\ltlf is not as expressive as \fin{RE}.
\ltlf is as expressive as FO and star-free \fin{RE}.

We would like \fin{RE}'s expressivity.
We can't use \fin{RE} directly (nonelementary nonemptiness of star-free \fin{RE} $\Rightarrow$ nonelementary complementation).
Thus we use \ldlf which can be easily complemented and intersected.
\ldlf can be translated into \fin{RE} elementarily.
Note this means \fin{RE} can't be translated into \fin{RE} elementarily otherwise we'd have elementary complementation for REf.
(Discussed more in the section on \ldlf)
 % section on LTL



\begin{comment}
Note to writer:
Important commands: (math mode)
\di{x} writes x in diamond brackets
\sq{x} '' in square brackets
\prog  symbol for program, currently PI
\vphi  alternate phi symbol (curly)
\model symbol for a PDL/modal logic model
\neg   logical negation symbol
\end{comment}

\section {Linear Dynamic Logic}

Linear Dynamic Logic (LDL) is exactly Propositional Dynamic Logic (PDL)\cite{dynamic-logic} in syntax. However, its semantics or models are based on finite traces, like LTL. Thus to explain it we will take a detour through PDL.


\subsection{Propositional Dynamic Logic}

We will describe the syntax and semantics of PDL piecewise, and then put it all together at the end of this section.

In PDL there are two entities, \emph{formulae} and \emph{programs}, which are mutually recursive. ``Programs'' are embedded into formulae by a \emph{modality}, and formulae into programs by a ``test construct'' which is like an assertion in programming.

\subsubsection{PDL Programs}

Now what is the sense of ``program'' here? There are no assignable variables as in normal programming-- this would require a domain, e.g. Int, Bool, Char, to assign from-- this is, in other words, a \emph{first-order} concept. We are dealing with \emph{propositional} logic here, so programs are considered at their most basic: as \emph{binary relations} between states.
\begin{comment}
In functional programming, programs are thought of as functions; this more general view allows for nondeterminism and partial functions.
\end{comment}

So, one of the things a \emph{model} $\model$ for PDL with \textit{atomic programs in the set $\prog_0$} must specify is a set of states $W$, and binary relations over the states that \textit{interpret} the atomic programs.

For example, with $\prog_0 = \set{x, y, z}$, we might have a model $\model$ with  $W = \set{1,2,3}$, and $x_{\model} = \set{(1,3)}, y_{\model} = \set{(3,2)}, z_{\model} = \set{(1,2),(2,3),(1,3)}$.

\begin{tikzpicture}[shorten >=1pt,node distance=2cm,on grid,auto]
   \node[state] (q1)   {$1$};
   \node[state] (q2) [above right=of q1] {$2$};
   \node[state] (q3) [below right=of q2] {$3$};
    \path[->]
    (q1.340) edge node {x, z} (q3.200)
    (q1)     edge node {z} (q2)
    (q2.335) edge node {z} (q3.115)
    (q3) edge node {y} (q2);
\end{tikzpicture}

Thus, executing program x at state 1, we know we will end up at state 3. Executing z at state 1, we may end up in state 2 or state 3. Just having atomic programs is a bit restrictive though-- we know we can get from 1 to 3 with one application of $x$, and 3 to 2 with $y$, but what if we can use $x$ \emph{or} $y$ in one step, or $x$ then $y$? Or, where can we get to by repeated applications of $z$? What we want is to be able to form the \emph{union}, \emph{composition}, and \emph{reflexive transitive closure}, respectively, of those relations. These correspond syntactically to the regular expression operators of \emph{addition} ($+$), \emph{concatenation} ($;$), and \emph{star} ($*$). Thus part of the syntactic definition for programs is:

\begin{myGrammar}
$\prog$: $\prog_0$; $\prog$ + $\prog$; $\prog$ \semi $\prog$; $\prog^*$
\end{myGrammar}


\noindent where $\prog_0$ is the atomic program set. We have left out the test construct for now, so the definition so far is \emph{test-free}.

\subsubsection{PDL Formulae}

Now we consider \emph{formulae}. PDL is a \emph{modal} logic, so formulae will be evaluated \emph{at specific states} in a given model. As we are dealing with propositional logic we will have the usual syntax

\begin{myGrammar}
$\vphi$: $\A$; $\neg \vphi$; $\vphi \lor \vphi$
\end{myGrammar}

\noindent where $\A$ is a proposition from the set of propositions $\prop$. Each state in the model must then have a propositional assignment over $\prop$. Alternatively, the model has a valuation $V$ such that $V(\A)(w)$ iff the proposition $\A$ holds at the state $w$.

Additionally, we can add programs through modalities, with formulae of the form \begin{myGrammarPlus} $\vphi$: $\di{\prog}\vphi$. \end{myGrammarPlus} The semantics of this is that the formula $\di{x}\vphi$ holds at a state $u$ in a model $\model$, if and only if $\vphi$ holds at a state $v$ in $\model$ such that $u\ x_\model\ v$. That is, if `we can reach a state via $x$ where $\vphi$ holds'.

The dual, $\sq{x}\vphi := \neg\di{x}\neg\vphi$, holds if $\vphi$ holds at all states reachable via $x$.

Now we can complete the loop by embedding formulae into programs: the test construct \begin{myGrammarPlus} $\prog$: $\vphi?$ \end{myGrammarPlus} is interpreted as a relation that relates a state $w$ \emph{to itself} if and only if $\vphi$ holds at $w$. So, if it doesn't hold, the program $\vphi?$, and indeed any concatenated program $\vphi?x$, will reach no states from $w$.

Thus we may express such programming constructs as



$\<if>\ \vphi \<then>\ x \<else>\ y$, which is $(\vphi?x)+(\neg\vphi?y)$ (omitting the ; for concatenation of programs); and

$\<while>\ \vphi \<do>\ x$, which is $(\vphi?x)^*;\neg\vphi?$ (remembering that a while loop terminates only when $\neg\vphi$ holds).

Additionally, we can express the Hoare triple $\set{\vphi} x \set{\psi}$ as $\vphi \to \sq{x} \psi$. (If $\vphi$ holds, then after all executions of $x$, $\psi$ holds. Note that this version does not enforce termination, since $\sq{x}\vphi$ holds at any state that reaches nowhere via $x$.)

\subsubsection{Listing of syntax and semantics}
Here we list all the syntax and semantics of PDL for convenience.
\begin{framed}
\par \textbf{Syntax:}
\par Programs:
\begin{myGrammar}
$\prog$: $\prog_0$; $\vphi?$; $\prog$ + $\prog$; $\prog$ \semi $\prog$; $\prog^*$
\end{myGrammar}
\par Formulae:
\begin{myGrammar}
$\vphi$: $\A$; $\neg \vphi$; $\vphi \lor \vphi$; $\di{\prog}\vphi$
\end{myGrammar}
\par \textbf{Semantics:}
\par Models: $\model = (W,\set{x_\model}_{x \in \prog_0}, V)$ where $W$ is a set of states, $x_\model$ are the interpretations of each atomic program, and $V(A)(w)$ is true iff proposition $A$ holds at state $w$.

Interpretation of programs:
\begin{itemize}
\item Base cases: \[
  R_\model(x)=\begin{cases}
               x_\model \quad \text{if $x$ is an atomic program}\\
               \set{(w,w) | \model,w \models \vphi } \quad \text{if $x = \vphi?$}\\
            \end{cases}
\]

\item Regular operations:\\
\mbox{\inference[add(L)]{u\ x_\model\ v}
                        {u\ (x+y)_\model\ v}
$\quad$ \inference[add(R)]{u\ x_\model\ v}
                         {u\ (y+x)_\model\ v}
$\quad$ \inference[concat]{u\ x_\model\ v & v\ y_\model\ w}
                          {u\ (x;y)_\model\ w}
}
\\

\mbox{\inference[star(Reflexive)]{}{u\ (x^*)_\model\ u}$\quad$
      \inference[star(Transitive)]{u\ x_\model\ v & v\ (x^*)_\model\ w}
                            {u\ (x^*)_\model\ w}}
\end{itemize}
\par Entailment of formulae:
\begin{itemize}
\item $\model,w \models A$ iff $V(A)(w)$ for propositional $A$.
\item The cases for $\neg$ and $\lor$ are as usual, staying at the same state.
\item $\model,w \models \di{x}\vphi$ iff there exists $v$ such that $ w\ R_\model(x)\ v$ and $\model, v \models \vphi$
\end{itemize}
\end{framed}
\subsection{LDL in terms of PDL}

We may now understand LDL in terms of PDL.
In the paper \cite{ldlf}, programs ($\prog$) are referred to as regular expressions ($\rho$). The atomic programs, written $\phi$ (not to be confused with $\vphi$), are propositional formulae over $\prop$ ($\bool{\prop}$) (a formula $\phi$ would represent the set of $\prop$-assignments which satisfy it, so $\bool{\prop}$ is finite up to this notion). We mentioned earlier that LDL models are a subset of PDL models.

Formally,

\begin{defnL}{ldl-model}
An LDL model is a finite trace $\pi=\pi[0]...\pi[last]$ of length $n$, where $last = n-1$, and each $\pi[i]$ is (labelled with) a propositional assignment over $\prop$ (alternatively, is a member of $2^\prop$).
\end{defnL}

As a PDL model, it would be $\model = (W, \set{\phi_\model}_{\bool{\prop}}, V)$ where
\begin{itemize}
\item $W = \set{\pi[i]\ |\ 0 \leq i \leq last}$,
\item $\phi_\model = \set{(i,i+i)\ |\ 0 \leq i \leq (last-1),\ \pi[i] \models \phi}$, and
\item $V(A)(\pi[i])$ iff $\pi[i](A)$.
\end{itemize}

If we had a ``next'' relation $next=\set{(i,i+1)\ |\ 0 \leq i \leq (last - 1)}$, we could think of $\phi_\model$ as $(\phi?next)_\model$ (considering $\phi$ as a formula $\vphi$ rather than a regular expression $\rho$). As it is, we can define the LTL operator $\circ$, which is interpreted as $next$, by $\circ:=\di{true}$.

\subsection{Examples of LDL formulae and models}

\begin{itemize}
    \item $\di{(\vphi_1?;true)^*}\vphi_2$ is the equivalent of the LTL formula $\vphi_1 \luntil \vphi_2$.
    \item $\sq{c}\sq{a+b}true$ would be satisfied by (among others) any model starting with $\pi[0]$ a set \emph{not} containing $c$. The point $\pi[0]$ would in this case not have any outgoing $c$-edges, and hence the condition `` for all points $\pi[i]$ reachable by a $c$-edge, $\pi[i] \models \sq{a+b}true$ '' (which is the condition for $\sq{c}\sq{a+b}true$ to be satisfied at $\pi[0]$) is satisfied vacuously.

\end{itemize}

In the next section we will describe alternating finite automata, which are generalisations of nondeterministic finite automata that are convenient to use to decide the satisfiability problem for LDL on finite traces.
 % section on LDL


\section {Alternating Finite Automata}

The class of regular languages (of finite words) is very robust.
Not only is it captured by the formalisms of deterministic and non-deterministic
finite automata (DFAs and NFAs) and regular expressions (RE),
but also by monadic second order logic (MSO),
various combinatorial games, and another type of finite automaton,
the alternating finite automaton (AFA).
In this section we will discuss AFAs.
First we recall the formal definitions of DFAs and NFAs.
Actually there are many variations on the definitions.
Some have sets of initial states. Some have epsilon-transitions.=
And AFAs can be viewed in many different ways.
So it is good to specify precisely which we are talking about.

\break

\subsection{DFA}

\begin{defnL}{dfa}
    A DFA is a 5-tuple $(\Sigma, Q, q_0, \delta, F)$, where $\Sigma$ is a finite set of symbols, $Q$ a finite set of states, $q_0$ a state in $Q$, transition function $\delta: Q \times \Sigma \to Q$, and final states $F \subseteq Q$.
\end{defnL}

A run of a DFA $(\Sigma, Q, q_0, \delta, F)$ on a word $w=w_0...w_{n-1}$ is a sequence of states $r_0...r_n$ where

\rundef{$r_0 = q_0$}
       { $r_{i+1} = \delta(r_i, w_i)$ for $i=0...n-1$}
       {$r_n \in F$}

A DFA accepts a word $w$ iff there is an accepting run on $w$. Note that it is simple to find out if there is an accepting run, since there is only one run on any word $w$ from any particular state $q$; this is the meaning of deterministic.

\subsection{NFA}
An NFA is exactly the same as a DFA except for the transition function, which is of type \\ $\delta: Q \times \Sigma \to \powerset{Q}$, and a run accordingly differs only in the transition condition which becomes $r_{i+1} \in \delta(r_i, w_i)$ for $i=0...n-1$. Again, a word is accepted by the NFA if there is an accepting run on the word. Here there may be multiple runs for any particular word, so acceptance is not as trivial to determine as for DFAs.

\noindent Note that a DFA can easily be `lifted' to an NFA thus:

$\mathit{liftd2n} (\Sigma, Q, q_0, \delta, F) = (\Sigma, Q, q_0, \Delta, F)$ where $\Delta(q, a) = \{\delta(q,a)\}$.

\noindent Moreover, in the other direction, an NFA can be determinised by the power set construction.

Some definitions of NFAs have $\epsilon$-transitions, but these can be factored out. Consider $N_1 = (\Sigma, Q, q_0, \delta, F), \delta:  Q \times \Sigma_\epsilon \to \powerset{Q}$ where $\Sigma_\epsilon = \Sigma \cup \set{\epsilon}$. Then we define $N_2 = (\Sigma, Q' = Q \cup q_0', q_0', \delta', F')$, $\delta:  Q' \times \Sigma \to \powerset{Q'}$ thus:
Firstly, for every state $q \in Q$ and non-$\epsilon$ symbol $a$, we want the $\delta'$ to reach all states which could be reached via $\epsilon$ transitions and one $a$-transition by $\delta$. If we define $E(S): \powerset{Q} \to \powerset{Q} =$ all states reachable from a state in $S$ by 0 or more $\epsilon$-transitions, then
$\delta'(q,a) = E(\bigcup_{r: E({q})} \delta(r,a))$. The new start state is $q_0' = E(q_0)$, and the transition function on it is $\delta'(q_0',a) = E(\bigcup_{r: E(q_0')} \delta(r,a)) $. $q_0'$ is in $F'$ if it has a member which is also in $F$, as is any state in $Q$ $q$ such that $E(q)$ has a member in $F$.


\subsection{AFA}


A hint as to what an AFA is like is that an NFA can be lifted to an AFA thus:
$\mathit{liftn2a} (\Sigma, Q, q_0, \delta, F) = (\Sigma, Q, q_0, \Delta, F)$ where $\Delta(q, a) = \bigvee_{r \in \delta(q,a)} r$.
\noindent For example, an NFA at state $q$, reading a word $av$, where $\delta(q,a) = \set{r, s, t}$ accepts only if it accepts the rest of the word $v$ from $r$, \textit{or} from $s$, \textit{or} from $t$. That is what the $\bigvee$ is about. In an AFA, we also have $\land$ as well as $\lor$, and as well as or, for all as well as exists, e.g. an AFA at state $q$ reading a word $av$ might accept only if it accepts the rest of the word $v$ (\textit{both} from $r$ \textit{and} from $s$), \textit{or} from $t$ (in this case $\delta(q,a) = (r \land s) \lor t$).

The formal definition is as follows:
\begin{defn} \label{defn:afa}
an AFA is the same as an NFA or DFA, except the transition function is of type $\delta: Q \times \Sigma \to \posbool{Q}$, \end{defn}


\noindent where $\posbool{Q}$ denotes propositional formulas without negation (positive), using states in $Q$ as propositions (note that $\posbool{Q}$ also includes formulas \textit{true} and \textit{false}). The AFA is in some sense a 'negation normal form' (nnf) logical formula. This idea is elaborated on in the section on LDL, in particular the translation from LDL to AFAs, which uses nnf.

We note the following property of $\posbool{Q}$, the monotonicity property:

\newcommand{\assA}{\mathcal{A}}
\newcommand{\assB}{\mathcal{B}}
\begin{lem} \label{lem:monotonic-posbool}
For any positive formula $F \in \posbool{Q}$, and for any assignments $\assA$ and $\assB$ such that $\assA \leq \assB$, $\assA \models F$ implies $\assB \models F$. (Contrapositively, $\assB \not\models F$ implies $\assA \not\models F$)
\end{lem}
\newcommand{\assC}{\mathcal{C}}
\textbf{Proof}: By induction on structure of the formula:
\begin{itemize}
\item Base case ($true$ and $false$): Trivial, since all assignments satisfy $true$ and none $false$
\item Base case (positive literal $q$): If $\assA \models q$, and $\assA \leq \assB$, then $\assB \models q$ by definition of $\leq$.
\item Inductive case ($\lor$): consider $F \lor G$ and $\assA \models F \lor G$. Then $\assA \models F$ or $\assA \models G$. In the first case, then by the inductive hypothesis any $\assB$ such that $\assA \leq \assB$ satisfies $F$, and so satisfies $F \lor G$. Similarly for the second case.
\item Inductive case ($\land$): consider $F \land G$ and $\assA \models F \land G$. Then $\assA \models F$ and $\assA \models G$. Then by the inductive hypothesis any $\assB$ such that $\assA \leq \assB$ satisfies $F$ and satisfies $G$, and so satisfies $F \land G$.
\end{itemize}

\begin{comment}
We also have a dual for negative formulae:
\begin{lem}
For any negative formula $F \in \negbool{Q}$, and for any assignments $\assA$ and $\assB$ such that $\assA \geq \assB$, $\assA \models F$ implies $\assB \models F$. (Contrapositively, $\assB \not\models F$ implies $\assA \not\models F$)
\end{lem}

Using both, we can restore the use of $\neg$, (if we adjust our definition of positive and negative formulae) which was actually implicit in $\land$ and $\lor$.
Suppose $\neg F$ is a negative formula, and we know $\assA \models \neg F$. Then $\assA \not\models F$, and $F$ is a positive formula. Using \ref{lem:monotonic-posbool} contrapositively, for any $\assC$ such that $\assC \leq \assA$, $\assC \not\models F$, so $\assC \models \neg F$.
\end{comment}

\subsubsection{AFA runs}
An AFA accepts a word $w$ if there is an accepting run on the word. The definition of a run might seem a bit involved at first-- it will soon be explained in more detail.

\begin{defnL}{subset-run}

    A run on a word $w=w_0...w_{n-1}$ is now a sequence of subsets of $Q$,  $S_0...S_n$, rather than just a sequence of elements of $Q$. A run must satisfy the following conditions:
    \rundef{$S_0 = \set{q_0}$.}
           {$S_{i+1} \models \delta(q, w_i)$ for all $q \in S_i$, $i=0...n-1$ \\ The notation $S \models f$ ($S$ satisfies $f$) means: The set $S$ is interpreted as the propositional assignment of 1 to any state in $S$, and 0 to all those in $Q \backslash S$.}
           {$S_n \subseteq F$. \\ A run is accepting if this accepting condition holds.}
\end{defnL}

A few notes: For the transition condition, it is possible that there is no set $S_{i+1}$ that satisfies every $\delta(q,w_i)$, which occurs exactly when some $\delta(q,w_i)$ is unsatisfiable (is equivalent to $false$). In that case the entire run fails there. On the other hand, if some $\delta(q,w_i) = true$, then it ``adds no burden'' to $S_{i+1}$, in that even the empty set would satisfy it-- it is as if we didn't have to consider it at all. \\ Also note that if $S_n$ is the empty set, it trivially satisfies the accepting condition, and that if for any $i$, $S_i$ is empty, then the transition condition allows all the sets following it to also be empty.

To make sense of the definition of run, consider an alternative definition of accepting a word, which follows the intuition explained in comparing NFAs to AFAs, formalised by logical formulae (I read about this in \cite{kumar}): For a state $q$ and word $w$ we inductively define a formula $L(q,w)$ which evaluates to 1 ($L(q,w) \evalto 1$) iff the AFA accepts $w$ starting from $q$. The base case is $w=\epsilon$. This is simple: if there are no more symbols remaining, all we have to do to determine acceptance is check whether the state is final or not, viz. $L(q,\epsilon) = q \in F$.
Otherwise, $w=av$, where $a \in \Sigma$. Then we take the boolean formula $f$ defined by the transition function $\delta(q, a)$, and replace the propositions/states in $f$ with the appropriate logical formula, which represents whether the AFA accepts $v$ from that state, viz. $L(q,av) = \delta(q, a)[L(r,v) / r$ for $r \in Q]$.

Then we can explain an accepting run thus: it is a proof that $L(q_0,w) \evalto 1$. The proof goes backwards along the run. We want to show by induction from $i=n$ to $0$ that $L(q,w_{-i}) \evalto 1$ for all $q \in s_i$, where $w_{-i}$ is in Haskell terms $\mathit{drop}$ $i$ $w$. %, and $L'(k,w) =\begin{cases} L(k,w) & k \in Q \\ k & otherwise \end{cases}$ .
\begin{itemize}
\item By the accepting condition, the case $i=n$ is trivial.
\item Supposing the IH is true for some $i=j \geq 1$, then we show it is true for $i=j-1$:
\\ Consider an arbitrary $q \in s_{j-1}$. We want to show that $L(q,av) \evalto 1 $, where $av = w_{-(j-1)}$. By the transition condition, $s_j \models \delta(q,a)$, so any assignment $\geq s_j$ will also satisfy $\delta(q,a)$ (by lemma \ref{lem:monotonic-posbool}). Now by definition, $L(q,av) = \delta(q, a)[L(r,v) / r$ for $r \in Q]$ and $eval \circ [L(r,v) / r$ for $r \in Q]$ is just such an assignment, by the IH for $j$ ($L(r,v) \evalto 1$ for $r \in s_j$). Thus $\delta(q,a)[L(r,v) / r$ for $r \in Q] \evalto 1$ and so $L(q,av) \evalto 1$.

\end{itemize}

We have shown that $L(q,w_{0}) \evalto 1 $ for all $q \in s_0$. By the initial condition $s_0 = \set{q_0}$ and $w_{0} = w$, so $L(q_0,w) \evalto 1 $, as required.

Note that the notion of `for all' is in the IH that $L(q,w_{-i}) \evalto 1$ \textit{for all} $q \in s_i$, while the notion of `exists' is that a word is accepted if \textit{there exists} an accepting run on the word.



\subsubsection{AFA to NFA}

Now we can show how to convert any AFA into an equivalent NFA, and as in the case of NFA to DFA, it is done by a subset-construction. The NFA simulates runs of the AFA, which are sequences of non-empty subsets of $Q$. The `for all' aspect is captured by these subsets and the transition function, while the `exists' aspect is part of the operation of NFAs.


$\mathit{afa2nfa} (\Sigma, Q, q_0, \delta, F) = (\Sigma,\powerset{Q}, \set{q_0}, \Delta, \powerset{F}) $ where \\ $\Delta(s, a) = \{ s' \subseteq Q | s' \models \delta(q,a)$ for all $q \in s \}$

Note that the definition of $\Delta$ is a bit inefficient since we only really need the minimal subsets of $Q$ which satisfy all the transition formulae.

Interestingly a DFA formed from this NFA would then have as states sets of sets of states of the AFA, which could be thought of as DNF formulas...

$\mathit{nfa2dfa} (\Sigma, Q, q_0, \delta, F) = (\Sigma,\powerset{Q}, \set{q_0}, \Delta, \set{S \subseteq Q | S \cap F \neq \emptyset}) $ where \\ $\Delta(s, a) = \bigcup_{r \in s} \delta(r,a)$

\subsubsection{Other Definitions of AFA}

There are quite a few other definitions of AFA. One uses trees to represent runs rather than a sequence of states. Another uses two disjoint sets of states (one 'forall' set and one 'exists' set). This leads naturally to a view of acceptance as a winning strategy in a game.

For now we will just present tree-runs.

\begin{defnL}{tree-run}
    A run is a tree in which the nodes are labelled by states, satisfying the following:
    \rundef{The root node is (labelled by) $q_0$.}
           {The (labels of) children of a node (labelled by) $q$ at level $i$ form a set $s \models \delta(q,w_i)$.}
           {Each branch must end either in a state which transitions to $true$ at any depth $\leq n$ or a final state at depth $n$.}
\end{defnL}
If we collapse all the nodes at each depth to a single set, we get the definition of run given earlier (\ref{defn:subset-run}) (Because the codomain of $\delta$ is \textit{positive} formulas, any superset of a set $s \models \delta(q,w_i)$ still satisfies $\delta(q,w_i)$). Conversely, each set in a run as sequence of subsets, which satisfies all $\delta(q,w_i)$ of the previous level $i$, can simply be duplicated to form the children of nodes $q$ in level $i$. The fact that the tree can be converted into a sequence of sets means it doesn't really matter what parent each node has to determine whether the AFA can accept the word from there. This is called the memory-less property.

\subsubsection{Computational Properties}

e.g. closure under language union, intersection, negation
AFAs cover the same language class as NFAs and DFAs; hence they are closed under all the regular operations,
as well as intersection, projection, and negation.
However, due to their particular shape, they are more \emph{effectively} closed under union, intersection and negation,
but projection is more involved. As for concatenation... well considering $A_1$ and $A_2$,
with the language of $A_2$ not containing $\epsilon$ for now, then $A_1;A_2$ could be formed by...
having a disjunction in the transition from each final state of $A_1$ to the start state of $A_2$,
and considering the final states of $A_2$ as that of the whole.
If $L(A_2)$ did contain $\epsilon$, then we would simply include as final states those of $A_1$ as well.
In fact this construction should work directly for NFAs without $\epsilon$-transitions since it just uses disjunction.
Union and intersection are implemented by the $\land$ and $\lor$ operators,
while negation is implemented by 1. switching occurrences of $\land$ and $\lor$
(and true and false) in transition functions, and
2. flipping the final states ($F' = Q \setminus{F}$)
For this reason, one can see that the shape of AFAs is like a negation-normal form formula.
(This is exactly what one does to negate a negation-normal form formula)

\subsection{Translation from LDL into AFAs}

\red{STUB}
The translation from LDL to AFAs given in \cite{ldlf} is as follows:

...
firstly LDL must be put into negation normal form.
This can be done in linear time.

\par Programs:
\begin{myGrammar}
$\prog$: $\prog_0$; $\vphi?$; $\prog$ + $\prog$; $\prog$ \semi $\prog$; $\prog^*$
\end{myGrammar}
\par Formulae:
\begin{myGrammar}
$\vphi$: $\A$; $\neg\A$;  $\vphi \lor \vphi$; $\vphi \land \vphi$; $\di{\prog}\vphi$;  $\sq{\prog}\vphi$
\end{myGrammar}

The states of the automaton corresponding to $\vphi$ will be the
Fischer-Ladner closure $CL_\vphi$, a generalisation of subformula closure...
\red{Type out definition of Fischer-Ladner..}
The automaton produced will have no final states; acceptance is by
transition to ``true'' ($\delta(\vphi,\Pi)=true$).
The transition function is recursively defined over the structure of states
(which are formulae).
Due to the formulae being in nnf, there is some ``repetitiveness'' due to duality
in the definition.
 % section on AFAs


\section {Implementation of Satisfiability Algorithm}
\subsection{Description}

As for the practical part of the project, well.
It involved encoding all the above formalisms
and then straightforwardly writing in the translations.
The most difficult part was choosing the right encodings.

As an approximation to the problem I tried writing a Haskell program to decide satisfiability for LTL.

I wrote a haskell program to try to answer the satisfiability question for LTL.
I did this by translating LTL formulae into equivalent deterministic finite automata, using basic automata and combinators, and testing for emptiness. For example, I translated $\phi$ Until $\psi$ as:
\begin{code}
> ltl2aut (Until f1 f2) = (dStar \$ ltl2aut f1) `dConc` ltl2aut f2
\end{code}
i.e. $ ltl2aut(\phi)^* \cdot ltl2aut(\psi) $.
But this was an error (I copied the translation of LTL to LDL in \cite{ldlf} which incorrectly translated $ f(\phi \, \mathcal{U} \, \psi) $ to $\langle f(\phi)^* \rangle f(\psi)$).
The correct translation of $ f(\phi \, \mathcal{U} \, \psi) $ into LDL is $\langle (f(\phi)?true)^* \rangle f(\psi)$. It doesn't seem simple to translate this directly to a DFA. So it makes sense to translate it (through LDL) to an AFA, as shown in \cite{ldlf}.

These were the issues that came up in coding:

The first was suitable representation of logics and automata.
In my first attempts the data types were not general enough
to cope with multiple logics and automata.
Particularly with doubly recursive types like LDLogic.
To solve this I made heavy use of Haskell's parameterised data types,
type synonyms and newtypes.
I decided it was not necessary to go all the way down to DFAs from
AFAs, and stopped at NFAs, since
emptiness of an AFA can be determined by looking at
the final state set of the corresponding NFA.

The second was the size of the automata.
The subset construction of an AFA produces an
NFA with exponentially many states.

As a partial solution, I only constructed reachable states.
But there will be cases for which reachability does not prune many states
(i.e. all states are reachable).
\red{not really sure what else I can do to make the whole satisfiability
procedure faster...}


I think there were some cases where Haskell got in the way, such as...
Hard to define "ShowSet" for arbitrarily nested sets.
Can't easily make predefined types into instances of stuff.
Need to do this newtype thing. Not extensible like Ruby.
Would have to edit original source code of predefined type.

Thus, all the pretty-printing is ad-hoc.

On the other hand, rigid though it can be,
it's nice to have that clean marking
of different types for different logics.
\red{a bit hand wavy}

\red{STUB}
Here is an example of my program...
the stages of data structures passed through on the way
to satisfiability.
do I need to construct this by hand?
\red{
Formula:
\\AFA
\\NFA
\\Empty?
}

\subsection{Evaluation and Comparison to existing tools}


\red{To do...}
Systems for managing automata, logic and games have been in development for quite some time, notably GOAL (and ...?).

I tried GOAL and here's how it compares with my system...

I would not expect my implementation to be as refined as such systems.

\subsection{Code listings}

First attempt


data structures


Second attempt


data structures
 % section on implementation

\section{Conclusion and Further Directions}

\begin{thebibliography}{9}
\bibitem{ldlf}
  de Giacomo and Vardi,
  Linear Temporal Logic and Linear Dynamic Logic on Finite Traces (IJCAI13)
\bibitem{ldlfsynth}
  de Giacomo and Vardi, Synthesis for LTL and LDL on Finite Traces (IJCAI15)
\bibitem{ldlfinsens}
  de Giacomo, Masellis and Montali, Reasoning on LTL on Finite Traces: Insensitivity to Infiniteness
\bibitem{decserflow}
    W.M.P. van der Aalst and M. Pesic, DecSerFlow: Towards a Truly Declarative Service Flow Language (Web Services and Formal Methods 2006)
\bibitem{nonelem_starfree}
    L. J. Stockmeyer and A. Meyer,
    Cosmological lower bound on the circuit complexity of a small problem in logic

\bibitem{kumar}
  K.N. Kumar,
  \href{http://www.cmi.ac.in/~kumar/words/lecture06a.pdf}{Lectures on Words (6a)}
\bibitem{dynamic-logic}
    D. Harel, D. Kozen, and J. Tiuryn. Dynamic Logic. MIT Press,
    2000.
\end{thebibliography}


\end{document}

\begin{comment}
%%% Other sections that could be added:
Abstract
Code Testing
Literature Review

\end{comment}
