
\section{Linear Temporal Logic on Finite Traces: Uses and Limitations}

\subsection{LTL on infinite traces: A logic for specifying behaviours of infinite processes.}
Historically LTL has been used in the infinite setting,
In the infinite setting LTL is concerned with concurrent processes which run indefinitely. By constraints on 'traces' of the observable/recorded behaviours of systems of these interacting processes, it can ensure/verify properties such as:

Safety:
Mutual exclusion (so that two processes don't try to access a resource e.g. a printer at the same time).
$\lalways(p_1 \to \neg p_2 \luntil \neg p_1)$ says one direction of mutual exclusion, viz. while $p_1$ is using the printer
$p_2$ may not.

Fairness:
After making a request (to use a printer) it is always fulfilled (eventually) and not blocked up by another user indefinitely.

Liveness:
The most basic liveness is, ``X happens''. So, $\leventually a$ says proposition $a$ will hold eventually.
Or, infinitely often: $\lalways \leventually a$. (modalities associate to the right.)
\subsection{Syntax and semantics}
\red{STUB}
The syntax of LTL is propositional logic over a given set of propositions $\prop$ with modalities ``next'' and ``until'': $\lnext \phi$ and $\phi \luntil \psi$.
The (syntactic) abbreviations  $\leventually \phi := true \luntil \phi$ and $\lalways \phi := \neg \leventually \neg \phi$, which have an ``obvious'' semantics of ``eventually'' (or ``future'') and ``always'' (or ``globally'').
An infinite model for LTL is an infinite sequence $\pi$ (also called a trace) of propositional models
A sequence. A formula $\phi$ is satisfied by trace $\pi$ if it is satisfied from the beginning of the trace, that is, we say $\pi, 0 \models \phi$. (Note this is not the same as the propositional world $\pi[0]$ satisfying $\phi$!)
... basic propositional stuff... $\pi, i \models \lnext\phi$ iff $\pi, (i+1) \models\phi$... $\pi, i \models \phi\luntil\psi$ iff either $\pi, i \models \psi$ or ($\pi, i \models \phi$ and $\pi, (i+1) \models \phi\luntil\psi$.
The finite semantics is z. Finite sequence. What happens at the edge. Stuff mentioned in paper. Certain formulae equivalent. $\lalways \leventually \phi$ and $\leventually \lalways \phi.$
They differ in aspects w.
\red{would be nice to have a diagram of a trace}

\subsection{How the finite setting differs: planning and service process modelling}
\red{STUB}

Having considered in abstractum the differences of finite and infinite setting, we shall briefly discuss some applications of \ltlf.
One is ``one step down'' from the infinite process behaviour specification: of course, it is ``finite process behaviour specification'', which takes its major form in
``Business Process Modelling'', in which some service is to be rendered to a client, in the process of which many different agents will be involved
(hence this has some of the ``concurrent'' flavour of the infinite process case.)
The major formalism in this area is called DECLARE (see e.g. \cite{decserflow}), which ``compiles'' into \ltlf.

On the other hand, there is what seems to be an entirely different use, which is as a way to encode planning problems.
A formula is interpreted as a problem, and a trace as a possible plan solving the problem.
This involves a little bit of special encoding (which would be factored into the compilation from e.g. the major planning formalism,
PDDL 3.0, to \ltlf). This application is important so it will be worth understanding a little bit.

By nature a formula only talks about propositions that hold at any given ``state'' in a trace.
But planning about taking \emph{actions} to get from one state to another (at which certain conditions hold, which can be encoded as propositions).
We can think of this contrast as between \emph{node labels} and \emph{edge labels}
(visualising a trace as a sequence of nodes (states $\pi[i]$), and directed edges from each $\pi[i]$ (except the last) to the successor $\pi[i+1]$).
LTL does not have a notion of actions or edge labels.
Thus actions are encoded as propositions (which are after all the only things available to use).
There is a special set of propositions $\actions$, disjoint from the ordinary propositions $\prop$,
and certain conditions are imposed on them that any trace will have to satisfy, viz.
1. exactly one action proposition must hold at each state except the first (at which none),
representing the action taken to get to that state,
and 2. the action proposition at a state $\pi[i]$
and the other propositions from $\prop$ at $\pi[i-1]$ and $\pi[i]$ must
``behave properly'' according to the planning problem definition (that stipulate the conditions, encoded as $\prop$, which must hold before and after an action is taken).

\subsection{examples of use}

More on some notions which arise naturally in the settings mentioned above,
and their representation in their respective formalisms and thus in LTL.

The formalisms associated with planning and (terminating) process/service modelling are
PDDL 3.0.
In the DECLARE formalism are defined certain commonly used patterns such as:
Response, Succession, Precedence, Chain, Existence, ...
(in fact these are represented graphically)
which are encoded in \ltlf as:


\red{tangent?}
There is quite a strong connection between \ltlf and LTL on infinite traces,
since the question of whether a model satisfies an \ltlf formula can,
in many cases, be directly answered by using tools originally developed for LTL formulae on
\emph{infinite} traces (otherwise, the formula would have to be translated).
The formulae which can be thus handled are called `insensitive to infiniteness'.
\cite{ldlfinsens}
Many common patterns captured by \ltlf are such formulae.

e.g.

\subsection{variations on goals}
\ltlf can also capture temporal constraints while trying to reach the goal.
It can also capture temporally extended goals.

\subsection{limitations}
Procedural execution constraints cannot be expressed in \ltlf.
\cite{ldlfsynth} (i.e. programs)

Especially in the context of planning, we see a limitation when we want to express
``procedural execution constraints''. Essentially, when we want to have more structure in our actions;
when we want to have `compound actions', which are basically programs.
And this ties in very nicely with \ldlf since it is based on PDL which intrinsically has a notion of programs.
Although it's a bit awkward because \ldlf mimcs LTL in not really having edge labels, or rather it
gives up its edge labels for a sort of backwards-compatability with \ltlf, but then we need edge labels, so
we must go the old route of encoding actions as propositions...

Well, I suppose there are advantages to doing things this way, but I can't help but wonder
what would happen if a more direct representation of actions was allowed...
Well, one would want to keep linearity of the atomic actions,
and then the only question is whether the thing still compiles nicely into AFAs?

\subsection{Expressivity, complexity and the need for another formalism}

Unfortunately, LTL is not as expressive as one would like (on finite traces). For example, it cannot express capture regular properties such as a model being of even length, or..

These would be required for ?? `soft constraints'
examples being...

LDL has the same complexity for satisfiability etc., but is more expressive. It can handle those things mentioned above. (Is the same true on infinite?)

LTL is not as expressive as REf.
LTL is as expressive as FO and star-free REf.
We would like REf's expressivity.
We can't use REf directly (nonelementary nonemptiness of star-free REf $\Rightarrow$ nonelementary complementation).
Thus we use LDL which can be easily complemented and intersected.
LDL can be translated into REf elementarily.
Note this means REf can't be translated into REf elementarily otherwise we'd have elementary complementation for REf.
